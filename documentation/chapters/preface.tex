Dit project is gemaakt als eind-opdracht van de Minor: \quotes{Robotica, Domotica en Industriële automatisering} te Hogeschool Leiden. In dit verslag is vast gelegd wat we in de loop van het project hebben gemaakt en meegemaakt. Zowel bij dingen die goed als minder goed gingen staat vastgelegd hoe we dit hebben gedaan. Ook is er aangegeven hoe we het beter zouden doen in de reflectie.\\

Dit project hebben we uitgevoerd met Scrum in gedachte. We zouden werken op basis van sprints om het product zo snel mogelijk te realiseren en taken overzichtelijk te houden.\\

Het project is opgedeeld in vier delen en twee sprints:
\begin{enumerate}
    \item Het idee: Bedenken wat we willen realiseren in het project.
    \item Het onderzoek: Uitwerken wat we gaan maken en hoe.
    \item Iteratief ontwikkelen in sprints:
        \begin{enumerate}
            \item Sprint 1: Alle grote onderdelen van het project werken in brede lijnen.
            \item Sprint 2: Alle onderdelen worden bij elkaar gevoegd en afgesteld.
        \end{enumerate}
    \item Reflectie: Documenteren wat goed en fout ging.
\end{enumerate}

De sprints duren ongeveer elk twee weken. Voorgaand hebben we een onderzoek uitgevoerd om duidelijk te krijgen wat het was, dat we gingen maken.\\
