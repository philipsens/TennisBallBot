\subsection{Het idee}
Aan het begin van het project hadden we meerdere ideeën verzonnen. Zo hadden we verzonnen om een soort op afstand bestuurbare battle-bot te maken, die door middel van een livestream kon laten zien waar die was. Een ander idee was om een prullenbak naar je toe te laten rijden door met met object detectie een handgebaar te detecteren en hier op af te komen. Echter hadden deze ideeën wat problemen en hebben we voor een middenweg gekozen: Automatisch tennisballen verzamelen op een tennisveld met afstandsbediening voor extra besturing.

\subsection{Onderzoek}
Voorafgaand aan de sprint's hebben we een onderzoek uitgevoerd. Deze is uitgevoerd om er echter te komen \textit{wat} we gingen maken en \textit{hoe}. \\

We hebben het onderzoek gestart met een \quotes{Probleemanalyse} met daarin de \quotes{Aanleiding en context}, \quotes{Probleemstelling} en \quotes{Doelstelling}. Dit gaf ons een goed beeld over waarom we het onderzoek deden. In \quotes{Onderzoeksopzet \& Uitvoering} hebben we een hoofdvraag en deelvragen opgesteld waarvan we vonden dat deze ons dichter bij het gewenste eindresultaat zou brengen. De hoofdvraag luid: \textit{\quotes{Wat is de beste manier om een robot efficiënt en snel tennisballen op te laten rapen?}} De deelvragen zijn er op gericht om stapje voor stapje een beter antwoord op de hoofdvraag te krijgen. De vragen waren erg breed gesteld zodat we veel mogelijkheden open hielden. Dit vonden we belangrijk omdat we niet goed wisten in wat voor situatie de robot het beste gebruikt zo kunnen worden en hoe deze te werk zou gaan. Enige termen die onbekend zijn voor een leek werden in het \quotes{Theoretisch kader} toegelicht.\\

In het hoofdstuk \quotes{Resultaten} hebben we alle mogelijke antwoorden  -- die we konden bedenken -- op de vragen gegeven. Ook hebben we de vragen op het internet opgezocht en de bronnen bij de resultaten gemeld. Het maakte ons niet heel veel uit of de resultaten die we vonden haalbaar of praktisch waren, aangezien ze ons wel een beter beeld zouden kunnen geven voor de aanpak in het algemeen. De antwoorden die we bij de \quotes{Resultaten} hadden gevonden, hebben we vervolgens in de \quotes{Onderzoek \& discussie} besproken en beargumenteerd wat we de beste oplossingen vonden. Dit deden we door middel van bevindingen en eerdere ervaringen. Ook hebben we proeven gedaan om zeker te weten welke techniek goed en bruikbaar zou zijn. In de \quotes{Conclusie} vatten we samen wat we in de \quotes{Onderzoek \& discussie} hebben geconcludeerd. Ook is er uit op te nemen wat we zullen gaan doen om het product te realiseren.\\

Het onderzoek en de conclusie hiervan zitten als bijlage bij dit document.\\

