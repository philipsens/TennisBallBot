\textit{\quotes{Al sinds kinds af aan vond ik het leuk om apparatuur uit elkaar te halen. Oude harddisks en versterkers werden niet gespaard. Ik vond het fascineren om alle kleine onderdeeltjes samen te zien werken als één geheel. Later ben ik nog met elektronica blijven hobbyen, helaas niet zo veel als ik zou willen. Naast een microfoon die wat gesoldeerd moest worden en een scherm van een telefoon die vervangen moest worden, heb ik het nooit echt aangedurfd om iets technischere dingen te doen. Ik wilde hier graag meer van leren en mijn enthousiasme voor elektronica een nut geven. Om deze reden ben ik de minor gaan doen. Ik kijk er naar uit om binnenkort een oude VU-meter en Macintosh SE werkend te krijgen!}} -- Sergi Philipsen\\

\textit{\quotes{Mij leek de minor interessant omdat het te maken had met het schrijven en ontwikkelen van embedded software. Tevens was het ook een groot plus punt dat de eindopdracht van de minor een zelfgekozen opdracht zou zijn.}} -- Joey de Ruiter\\

Allereerst willen we Vincent Bakker en Herman van Haagen bedanken voor de lessen en begeleidingen die ze ons gegeven hebben. We kunnen ons voorstellen dat het als docent erg lastig is om kennis over te dragen op het moment dat de leerlingen geen directe feedback in de klas kunnen geven. Ondanks het werken vanaf thuis denken we dat het ze goed gelukt is om ons nieuwe dingen te leren en ons rijker van kennis te maken. Hiernaast willen we ook Justin Verkade bedanken voor het repareren van een elektromotor en Timo Kleverlaan voor het printen van een 3D model. Zonder hun hadden we niet tot dit mooie resultaat kunnen komen.
